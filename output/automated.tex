\documentclass[12pt,onside,a4paper,article]{memoir}
\title{Example Document}
% \author{}

\usepackage{amsmath, amssymb, amsthm, mathrsfs}
\usepackage[pdftex]{graphicx,color}
\usepackage[ngerman]{babel} 
\usepackage{nicefrac}
\usepackage{algorithm,algorithmicx,algpseudocode,MnSymbol} 
\usepackage{framed}\definecolor{shadecolor}{rgb}{0.9,0.9,0.9}
\usepackage[utf8]{inputenc} % set input encoding to utf8
\usepackage{afterpage}
\usepackage[margin=2cm,bmargin=5cm]{geometry}

\graphicspath{{/home/adi/}}
%\graphicspath{{/home/adi/Dokumente/Unterricht/Physiologie/Media/}}

\newcommand\blankpage{%
    \null
    \thispagestyle{empty}%
    \addtocounter{page}{-1}%
    \newpage}

% front matter %
\begin{document}

\title{Prüfung \protect\\ 
in Physiologie}
\date{16. Juli 2018}
\maketitle

\begin{shaded}
Name: 
\end{shaded}

Es sind keinerlei Vorlesungsunterlagen zugelassen. Wenn Sie eine Frage nicht verstehen, geben Sie ein Handzeichen und fragen Sie die Aufsichtsperson.\\
Die Prüfungsdauer beträgt 90 Minuten.

\begin{center}

\begin{tabular}{lc}
PAKE &  \hspace*{1cm}/4,0 Pkt.
\\
SSO und Kerberos   & \hspace*{1cm}/4,5 Pkt.
\\
TLS   & \hspace*{1cm}/5,5 Pkt.
\\
Bitcoin &  \hspace*{1cm}/5,0 Pkt.
\\\hline
Gesamt &  \hspace*{1cm}/19,0 Pkt.
\end{tabular}

\vspace*{0.5cm}
\begin{table}[b]
\begin{flushright}
\small
\begin{tabular}{|c|c|c|}
\hline
\multicolumn{3}{ |c| }{Notenschlüssel}
\\\hline
von (inkl.) & bis (exkl.) & Note
\\\hline
90\%  &  -- & Sehr gut (S1)
\\
80\%  & 90\% & Gut (U2)
\\
70\%  & 80\% & Befriedigend (B3)
\\
60\%  & 70\% & Genügend (G4)
\\
-- & 60\% & Nicht Genügend (N5)
\\\hline\hline
\multicolumn{2}{ |l| }{$\sum$ Punkte: }  & \\
\multicolumn{2}{ |l| }{ }&
\\\hline
\end{tabular}
\end{flushright}
\end{table}
\end{center}

\newpage
%\pagestyle{empty}
% \setlength{\parskip}{5mm}

\begin{enumerate}\itemsep30mm


% set the page numbers to be arabic, starting at page 1 %
\pagenumbering{arabic}
%\setcounter{page}{300}


\item{What color is a big red apple?}
\begin{flushright}
\scalebox{1.7}{
\begin{tabular}{|m{0.5cm}|m{0.5cm}|}
\hline
\ 1 &  \\ 
 \hline
\end{tabular}}
\end{flushright}
\item{Which continent do you live in?}
\begin{flushright}
\scalebox{1.7}{
\begin{tabular}{|m{0.5cm}|m{0.5cm}|}
\hline
\ 2 &  \\ 
 \hline
\end{tabular}}
\end{flushright}
\item{Is summer or winer warmer?}
\begin{flushright}
\scalebox{1.7}{
\begin{tabular}{|m{0.5cm}|m{0.5cm}|}
\hline
\ 3 &  \\ 
 \hline
\end{tabular}}
\end{flushright}
\item{What is your favourite programming language?}
\begin{flushright}
\scalebox{1.7}{
\begin{tabular}{|m{0.5cm}|m{0.5cm}|}
\hline
\ 4 &  \\ 
 \hline
\end{tabular}}
\end{flushright}
\item{How do you write documents?}
\begin{flushright}
\scalebox{1.7}{
\begin{tabular}{|m{0.5cm}|m{0.5cm}|}
\hline
\ 5 &  \\ 
 \hline
\end{tabular}}
\end{flushright}
\end{enumerate}

\end{document}